\documentclass[a4paper, 12pt] {article}
%this is a comment
\usepackage{color}
\usepackage{amsmath}
\usepackage{hyperref}
\usepackage{graphicx}
\begin{document}
\title{hola}
\date{$12^{th}$ November, 2014}
\author{aixa\\
learning center,\\
LG,\\
Uni of Birmingham\\
aixa@gr}
\maketitle
\newpage
\begin{abstract}my first latx gggggggggggg
\end{abstract}
\newpage
\tableofcontents
\newpage
\part{jjjjjjj}
\section{text and characters}
\subsection{bold}

\subsection{italics}
\subsection{texit}
\textit{jjj}
\subsection{emph}
\emph{ggg}
\subsection{typewritter}
\texttt{jjjj}
\subsection{underline}
ggggggg\underline{jjj}
\subsection{colors}
\begin{color}
{red}{jjj}
\end{color}
\subsection{lists}
\subsubsection{itemlist}
\begin{itemize}

\item{b}
{\setlength\itemindent{25pt}}
\item{f}

\end{itemize}
\begin{enumerate}
\item{b}
\item{f}
\end{enumerate}

\section{mahs}
the forst way to write maths in \LaTeX is in-line like this:
$a^2+b^2+c^2$
\begin{equation}\label{ols}
\hat{\omega}=(x'x)^{-1}x'y
\end{equation}
\section{analysis}
we can use ols estimator shown in equation 
\ref{ols}
\begin{equation}
\lim_{n \to \infty}
\sum_{k=1}^n \frac{1}{k^2}
=\frac{\pi^2}{6}
\end{equation}
\begin{equation*}
\mathbf{x} = \left(
\begin{array}{ccc}
a & b & \ldots \\
b & c & \ldots \\
c & d & \ldots \\
\end{array}\right)
\end{equation*}



\section{table}
\begin{center}
\begin{table}[!h]
\caption{our first table}
\vspace{0.15in}
\begin{tabular}{c | c | c | c}
subject & difficulty & fun & use\\ \hline
econo & 7 & 8 & 7\\
econo & 7 & 8 & 7\\
econo & 7 & 8 & 7\\
econo & 7 & 8 & 7\\
\end{tabular}
\end{table}
\end{center}

\section{citation}
\cite{william}
\bibliographystyle{apalike}
\bibliography{mybib}

\end{document}